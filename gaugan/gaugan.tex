% Created 2020-01-10 Fri 09:12
% Intended LaTeX compiler: pdflatex
\documentclass[a4paper, dvipdfmx, 10pt]{article}
\usepackage[utf8]{inputenc}
\usepackage[T1]{fontenc}
\usepackage{graphicx}
\usepackage{grffile}
\usepackage{longtable}
\usepackage{wrapfig}
\usepackage{rotating}
\usepackage[normalem]{ulem}
\usepackage{amsmath}
\usepackage{textcomp}
\usepackage{amssymb}
\usepackage{capt-of}
\usepackage{hyperref}
\usepackage{minted}
\usepackage{amsmath, amssymb, bm}
\usepackage{graphics}
\usepackage{color}
\usepackage{times}
\usepackage{longtable}
\usepackage{minted}
\usepackage{fancyvrb}
\usepackage{indentfirst}
\usepackage{pxjahyper}
\usepackage[utf8]{inputenc}
\usepackage[backend=biber, bibencoding=utf8]{biblatex}
\usepackage[top=20truemm, bottom=25truemm, left=25truemm, right=25truemm]{geometry}
\hypersetup{colorlinks=false, pdfborder={0 0 0}}
\usepackage{ascmac}
\usepackage{algorithm}
\usepackage{algorithmic}
\addbibresource{./qareport.bib}
\author{MokkeMeguru}
\date{\textit{<2019-04-24 Wed>}}
\title{強力なNormalization手法、GauGAN (SPADE) を読む}
\begin{document}

\maketitle
\tableofcontents


\section{導入}
\label{sec:orgc1fdfab}
GauGAN [1] というのは 2019 年画像生成系の分野を賑わせた新たな Normalization (正規化) 手法についての論文で、NVIDIAの研究成果になります。website [2] でそのデモを体験することが出来、日本でも深層学習をやっている人たちが盛り上がっていたイメージです。\\
あと \textbf{\textbf{実装が PyTorch なので}} 大変読みやすく、バグがないです。\\
\begin{center}
\includegraphics[width=18cm]{./gaugan_image.png}
\end{center}

[1]: \href{https://arxiv.org/abs/1903.07291}{Semantic Image Synthesis with Spatially-Adaptive Normalization}\\

[2]: \href{https://github.com/NVlabs/SPADE}{Github}\\

\section{GauGAN の取り組む問題}
\label{sec:orgd0facab}
GauGAN が取り組むタスクは Semantic Image Synthesis というものです。こは簡単に言ってしまうと \textbf{\textbf{こんな感じの画像を作りたい}} というイメージから \textbf{\textbf{写真のような画像}} を生成することを指し、パッとWebの背景画像を作りたいとか、ゲームの画像を作りたいとかいった場面で役に立ちます。\\

本研究の Semantice Image Synthesis において、入力は概形を描いたセグメンテーション画像 (色分けした画像) であり、出力は写真のような画像、となっています。関連研究としては、例えば入力が単語であったりする場合があります。\\

本研究で用いたデータセットの一つとして \href{https://github.com/nightrome/cocostuff}{COCO-Stuff dataset} があります。これは写真データと、そのセグメンテーション画像がペアになっています。本来的には COCO-Stuff は写真データ -> セグメンテーション画像、というふうな学習を目的としたデータセットなのですが、本研究ではその逆を行なっている点に注意して下さい。\\

\begin{center}
\includegraphics[width=18cm]{./coco.png}
\end{center}

\section{Normalization とは何か}
\label{sec:org297cc13}
Normalization (正規化) とは雑に言うと \(\bm{x}\) の値域を \(\bm{x'}\) へ変換することを指します。 \textbf{\textbf{変換前後の次元的に言うと、 \(f: \mathbb{R}^{H \times W \times C} \rightarrow \mathbb{R}^{H \times W \times C}\) です}} 。\\
画像データ \(\mathbb{R}^{N \times H \times W \times C}\) (\(N\) : バッチサイズ、 \(H\) : 画像の高さ、 \(W\) : 画像の幅、 \(C\) : 画像のチャンネル ) を例にすると、次のようなものが例として挙げることが出来ます。\\
\begin{itemize}
\item BatchNormalization\\

\(C\)  について \(\mu = 0\) ,  \(\sigma = 1\) への正規化をします。このとき平均・分散の求め方は \(\mu_B = \cfrac{1}{N  H  W}\Sigma X_{n, h, w}\) \(\sigma^2_B = \cfrac{1}{N  H  W}\Sigma (X_{n, h, w} - \bar{X)}^2 \in \mathbb{R}^{C}\)  のようになります。(バッチサイズ \(N\) が存在していることに注意)\\

正規化の式について簡単に取り上げると、一枚画像 \(x\) の \(C\) 軸について \(\hat{x_i} = \cfrac{1}{\sigma_B}(x_i- \mu_B), x_i \in \mathbb{R}^{C}\) となります。(但し実装や性能向上のために、実際はもっと複雑な式が用いられます。)\\
\end{itemize}
\begin{itemize}
\item InstanceNormalization\\

\(C\) について  \(\mu = 0\) ,  \(\sigma = 1\) への正規化をします。但しこのときの平均・分散の求め方は \(\mu_I = \cfrac{1}{HW}\Sigma x_{h, w}\) \(\sigma^2_I  = \cfrac{1}{HW} \Sigma (x_{h, w} - \bar{x})^2\) のようになります。Batch Normalization が画像データ群全体で \(C\) が \(\mu=0, \sigma^2=1\) となるようにしているのに対して Instance Normalization が 一枚の画像について \(\mu=0, \sigma^2=1\) に正規化している点が主な違いです。\\
\item ActNorm\\

ActNorm は Glow[3] で提案された Normalization で、これは \textbf{\textbf{まず}} 初期バッチ \(X_B = {x_1, x_2, ..., x_n}\) について、次の式に従うようにして \(C\) について  \(\mu = 0\) ,  \(\sigma = 1\) への正規化を行います。\\

\(\mu_{init} = \cfrac{1}{NHW} \Sigma X_{n, h, w}, \sigma^2_{init} = \cfrac{1}{NHW} \Sigma (X_{n, h,w} - \bar{X})^2 \in \mathbb{R}^{C}\)\\

ActNorm の \(\mu_{init}, \sigma^2_{init}\)  は、初期バッチで初期化された後、特に \(C\) について正則化をするという制約をかけられず、ただの訓練パラメータとして用いられます。つまりこの Normalization は \(\mu = 0\) ,  \(\sigma = 1\) への変換ではないという点に注意して下さい。\\
\end{itemize}

\begin{center}
\includegraphics[width=15cm]{./normalization.png}
\end{center}

正規化について直感的な図を Group Normalization [4] から引用しましょう。例えば BatchNormalization の場合、青い部分がシュッとなって \(\mu_{B_i}, \sigma^2_{B_i}\) となります。これが \(C\) 個出来るので、 \(\mu_B \in \mathbb{R}^{C}\) です。Instance Normalization は同様に考えると \(\mathbb{R}^{C B}\) ですが、バッチ軸についてはまとめ上げられるので \(\mathbb{R}^{C}\) となります。\\


[3]: \href{https://arxiv.org/abs/1807.03039}{Glow: Generative Flow with Invertible 1x1 Convolutions}\\

[4]: \href{https://arxiv.org/pdf/1803.08494.pdf}{Group Normalization}\\
\section{SPADE (Spatially-Adaptive (DE)normalization)}
\label{sec:orgb1a6f9a}
GauGANでは Normalization について新たな手法 SPADE (Spatially-Adaptive (DE)normalization) を提案しました。\\
  特にこの Normalization は Conditional Normalization の一種としてみなされます。Conditional Normalization の関連手法としては Conditional Batch Normalization や AdaIN を挙げることが出来ます。これらは外部のデータを用いた手法であり、この手順は (1)  BatchNormalization などの手法で 平均 0, 分散 1 へ正規化を行い \(x\) を獲得 、(2) 外部のデータを用いてアフィン変換 \(ax + b\) を行う、というものになっています。先行研究でのアフィン変換のパラメータ \(a, b\) はベクトルであったりスカラーであったりいろいろですが、SPADE ではここにセグメンテーション画像を用いました。\\
\subsection{SPADE のコンセプト}
\label{sec:org7661f9a}
\begin{quote}
their normalization layers tend to “wash away” information contained in the input semantic masks.\\
--- quoted from page 2 line 1\\
\end{quote}

SPADEのコンセプトは、 \textbf{\textbf{BatchNormalization らが "wash away(洗い流す)"  する内容を復元する}} ということです。そして彼らは復元する情報源として \textbf{\textbf{セグメンテーション画像}} を使いました。\\

直感的な説明をしましょう。例えばセグメンテーション画像からハワイの海岸の画像を生成しようとするとき、海の部分と砂浜の部分を同じように平均 0,  分散 1 にされてしまうと (ここで Batch Normalization が \(C\) について正規化されているという点を思い出してみましょう) 情報落ちてない?となるわけです。ここで Conditional Normalization をして情報補完をしてみよう→どうやって補完する?→そういえばセグメンテーション画像なんてものがあるな、みたいな感じに発想を進めていくことが出来ます(いや彼らがそう思っているかは知りませんが)。\\

下の画像が SPADE のレイヤーの概要です。確かに (1) BatchNormalization (2) セグメンテーション画像から \(\gamma, \beta\) を用いてアフィン変換、をしていますね。\\
\begin{center}
\includegraphics[width=10cm]{./spade_abst.png}
\end{center}

\subsection{SPADE の細かい話}
\label{sec:org55fb4c6}
次に細かい話としてSPADE の入力と出力を示しましょう。前提として、SPADE は BatchNormalization に合わせて モデルの複数ヶ所に適用されるので、それぞれの SPADE を \(i\) で区別します。\\

SPADEの入力はセグメンテーション画像で、セグメンテーションラベルは \href{https://github.com/NVlabs/SPADE/issues/29}{one-hot vector になっており} 、 つまりこれがいわゆるセグメンテーション画像の \(C\) になります。これが \(H^{i} \times W^{i}\) 個あるので、結局 SPADE の入力は \(m\in (\mathbb{L}^{H^{i}\times W^{i}} = \mathbb{R}^{H^{i} \times W^{i} \times C^{i}})\) となります。 \(\beta^{i}, \gamma^{i}\) をそれぞれ後述する式で求めます。\\
それはそれとして、BatchNormalization Layer から \(\mu^{i}, \sigma^{i}\) をそれぞれ用意しておきます。\\

BatchNormalization Layer に入ってくる Tensor \(h^{i}\) と \(\mu^{i}, \sigma^{i}, \beta^{i}, \gamma^{i}\) から \(h^{i}\) -> [BN -> SPADE] -> \({h^i}'\) は 次の式で表すことが出来ます。\\

\begin{eqnarray*}
{h^i}' = \gamma^i \cfrac{h^{i} - \mu^{i}}{\sigma^{i}} + \beta^{i}
\end{eqnarray*}

ここであれ?って思える人は深層学習の実装に向いています。そう、この式ですと次元数があんまりよくわかんないです。なので、詳しく次元数を書いてみます。\\

\begin{eqnarray*}
, where\\
h^{i} &\in& \mathbb{R}^{N \times H^{i} \times W^{i} \times C^{i}}\\ 
\mu^{i}, \sigma^{i} &\in& \mathbb{R}^{C^{i}} \\
 \gamma^{i}, \beta^{i} &\in& \mathbb{R}^{H^{i} \times W^{i} \times C^{i}}
\end{eqnarray*}

つまり Batch Normalization が \(C^{i}\) について正規化が行われているのに対して、SPADE は \(H^{i}, W^{i}, C^{i}\) について正規化が行われています。\\

雑な発想ですと、セグメンテーション画像をそのままペッと \(\cfrac{h^{i}-\mu^{i}}{\sigma^i}\) へ貼っているイメージでしょうか。こうすることで Batch Normalization で落としてしまったであろう情報を復元できるというわけです。\\
\section{モデルの全容}
\label{sec:org35608e2}
本研究で特にセグメンテーション画像を使っている部分は一般的なGANs でいう Generator と Discriminator の部分なので、これらについて概形→詳細、と詰めて見てみましょう。\\

\begin{center}
\includegraphics[width=10cm]{./gaugan_full.png}
\end{center}

\subsection{Generator}
\label{sec:org7f4156b}
最終的な出力が \(N \times H \times W \times C = 256 \times 512 \times 512 \times 3\) の画像となる Generator を下に引用します。\\

\texttt{SPADE ResBlk(K)} は入力を \(N \times H^{i} \times W^{i} \times C^{i}\) の入力を受け取り \(N \times H^{i} \times W^{i} \times K\) を出力します。そして \texttt{Upsample(2)} は \(N\times H^{i} \times W^{i} \times K\) を入力として \(N \times 2H^{i} \times 2W^{i}  \times K\) を出力とします。例として、上から1つ目の  \texttt{SPADE ResBlk(1024), Upsample(2)} は、 \(N \times 4 \times 4 \times 1024\) を入力として \(N \times 8 \times 8 \times 1024\) を出力とします。(以降バッチサイズ \(N\) を省略)\\
\begin{center}
\includegraphics[width=10cm]{./gaugan_gen.png}
\end{center}
\subsubsection{SPADE ResBlk}
\label{sec:orgc088769}
SPADE ResBlk については、下に引用される図で説明します。ResNet を元にしていますが、入力と出力の次元数が異なっている点に注意して下さい。簡単な構造は ResNet と変わっていませんが、正規化層がそのまま SPADE に置き換わっており、SPADEのための入力であるセグメンテーション画像が外部から与えられていることがわかると思います。\\
   \texttt{SxSConv-K} は カーネルサイズ \(S\) フィルタサイズ \(K\) の畳み込みを示しており、\(H\times W\times C\) のTensorを入力として \(H \times W \times K\) の Tensor を出力とします。\\
\begin{center}
\includegraphics[width=10cm]{./SPADE_ResBlk.png}
\end{center}

\subsubsection{SPADE}
\label{sec:org10c3c81}
SPADE そのものについては、下に引用される図で説明します。\\



\subsection{Discriminator}
\label{sec:orgf4b0964}

\section{訓練・推論手法}
\label{sec:org5ab6e20}
\section{実験}
\label{sec:org22621a9}
\section{結果}
\label{sec:org6710ff8}
\section{論文のアブストラクト・イントロダクションの和意訳}
\label{sec:orgf676c62}
アブストラクト\\
\section{読んだ感想とか}
\label{sec:org3eed845}
\end{document}
